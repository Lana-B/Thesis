\chapter{Introduction}
\label{c:intro}

The standard model (SM) of particle physics is the currently accepted theory for describing the known fundamental building blocks of the universe. It has stood up to rigorous testing at experiments such as DESY, LEP at CERN and the Tevatron at Fermilab. However there are still many unanswered question about the universe such as how do we integrate gravity and dark matter with the standard model to make a unified theory?
The Large Hadron Collider (LHC) at CERN aims to study possible signatures of new physics that may arise around the electroweak scale. One of the main goals of the LHC was to find the Higgs boson which was confirmed in July 2012. The precision measurement of standard model processes is also still very important as deviations from the SM expectation can give hints about new physics and they are also key background processes which should be well understood so that new physics signals can be found.
The SM process of the production of four top quarks is studied in this thesis. Although this process is predicted by the SM, it is extremely rare and hence it has not yet been possible to measure it. A precision measurement of four top quark production would be an exceptional test of the SM. In addition to this, many models of new physics predict final states which contain four top quarks, therefore the SM production of four top quarks is both an important background to these new models and the production rate of four top quarks may be enhanced by new physics models.

The thesis is structured as follows: First the background theory of particle physics and top quark physics is discussed in Chapter~\ref{c:theory}. The LHC and particularly the CMS detector are described in Chapter~\ref{c:det}. The reconstruction of physics objects from the detector read-out are given in Chapter~\ref{c:recon}. In Chapter~\ref{c:ana} the general strategy for searching for four top quarks and the analytical and statistical techniques required are described. The search for four top quarks in the single lepton channel in the 2012 dataset collected by the CMS experiment is described in Chapter~\ref{c:Run1}. The phenomenological interpretation of the former analysis in the context of a model of new physics were a sgluon particle is predicted is given in Chapter~\ref{c:pheno}. The search for four top quarks is continued in the 2015 dataset collected by the CMS experiment, described in Chapter~\ref{c:Run2}, where enhancements are made to the analysis and additional search channels, opposite-sign and same-sign dilepton are combined with the single lepton channel to produce a more sensitive final result. Finally a summary of the thesis can be found in Chapter~\ref{c:DandC}, including a discussion of the relevance of the analysis in the field of high energy physics and a look towards the future.

In this thesis the convention of using natural units, $\hbar = c = 1$, is adopted. 