\chapter{Event Reconstruction}
\label{c:recon}
% In this chapter the software and algorithms used to reconstruct particle physics objects are detailed. The idea is to work backwards from the information obtained from each of the sub-detectors to determine what particles passed through them.\\
% A software framework known as CMSSW has been developed in order to reconstruct the read out from the detector for each event


In chapter~\ref{c:det} each of the sub-detectors in CMS have been described; how particles interact with them and how electrical signals are read-out. The next step is to combine the read-outs from each detector in order to reconstruct the resulting particles from an interesting proton-proton collision. This snapshot of the collision output is known as an \emph{event}. An event will also contain reconstructed particles from other simultaneous uninteresting collisions from the same bunch crossing which is known as \emph{pile up} (PU). Algorithms are used in order to subtract PU particles from the stored event. 
As a particle will usually traverse more than one sub-detector, it is advantageous to combine these outputs in order to reconstruct the particle. This is achieved using the \emph{Particle flow} (PF) algorithm described in section~\ref{sec:PF}. The objects which can be reconstructed using the PF algorithm such as muons, electrons, and jets are discussed in sections~\ref{sec:muonreco},~\ref{sec:electronreco},~\ref{sec:jetreco} respectively. Further information can be obtained from these reconstructed objects such as how likely a jet is to have originated from a b-quark (section~\ref{sec:btagreco}) and how the presence of neutrinos can be inferred by the imbalance of energy in the transverse plane of the detector (section~\ref{sec:METreco}).  
The information from the detector is processed using a distributed computing infrustruction with the CMSSW software \fixme{reference}.
% This chapter will discuss the main software used to reconstruct events from the detector read-out, CMSSW, and how a worldwide computer farm is used to process data.


% \section{Software}

% To-do:
% CMSSW, grid

\section{Track reconstruction}

Approximately 1000 charged particles are expected to traverse the CMS tracker at each bunch crossing at a PU of $\approx 20$ concurrent collisions. Each particle will interact with the silicon tracker as it continues through it's trajectory from the collision point. Algorithms are designed to match hits in tracker along each particle's trajectory in order to reconstruct it's path so information about the charge and momentum of the particle can be obtained. Not only is the tracker information used in offline reconstruction but it is used in the HLT, therefore it must have a fast response. Reconstructed paths from random particle hits in the tracker are considered to be \emph{fake tracks}~\cite{1748-0221-9-10-P10009}.\\
Knowledge of particle trajectories can help to pinpoint the collision vertex also known as the \emph{primary vertex} and described further in section~\ref{sec:PVreco}. Accuracy in reconstructing tracks is essential for b-tagging as described in section~\ref{sec:btagreco}.
Electrons lose energy throught the tracker material in a non-gaussian way such that their tracks can not be fit using the standard Kalman Filter. A Gaussian-Sum-Filter refit, which uses a sum of gaussians to estimate the energy loss, takes into account the interaction of electrons through the tracker material.


\section{Primary vertices \label{sec:PVreco}}

Primary vertices are the point at which the collision occurred, as opposed the secondary vertices which originate at the decay of subsequest particles coming from the collision. The first step in reconstructing primary vertices is to consider tracks which are consistent with the beam spot and cluster them into candidate vertices, separated along the z direction. Next a 3D fit is made and candidates which are compatible with originating from the beamline are kept. Primary vertices are ranked according to the sum of the momentum squared of all the tracks considered to have originated from that vertex. The vertex with the largest sum is regarded as the signal vertex, ie. the event with higher momentum objects which is most likely to have fired the trigger. 

\section{Particle flow ~\label{sec:PF}}

The particle flow (PF) algorithm combines information from all sub-detectors, described in section~\ref{c:det}, in order to improve the reconstruction of final state particles such as electrons, muons, photons, neutral hadrons and charged hadrons. The biggest gains come from PF jet reconstruction where the jet-matching efficiency, jet energy resolution and the reconstruction of the jet \pt are improved compared to using calorimeter information only~\cite{CMS-PAS-PFT-10-001}. From this information more complicated objects can be reconstructed as described in the subsequent sections. Collections of different objectts are created and objects are subsequently removed from each collection as they are identified within the algorithm.\\

The particle flow algorithm reconstructs objects in an order starting from the easiest to reconstruct unambiguously. The hardest objects to reconstruct such as neutral hadrons are one of the last to be reconstructed because their properties can be constrained from the previously reconstructed objects.
The first objects to be reconstructed are PF muons. Each muon identified from the muon chambers is associated to the compatible hits in the tracker. This associated track is then removed from the collection. Next, a Gaussian-Sum-filter refit is used to extrapolate electron candidate trajectories to the ECAL. Electrons have relatively short trajectories due to losing most of their energy by Bremsstrahlung in the tracker material. Tracker and ECAL variables are combined for the final identification of a PF electron after which the track and ecal clusters are removed from there respective collections.

Charged hadrons are reconstructed from the remaining tracker, ECAL and HCAL deposits where the calorimeter hits are compatible with the tracker hits. Again, these hits are removed from the respective collections. Neutral hadrons leave no tracks in the tracker but have deposits in the ECAL and HCAL. Photons leave deposits in the ECAL but not the HCAL.

\section{Isolation \label{sec:isolation}}

Relative isolation (\emph{RelIso}), is a measure of how isolated the muons or electrons are from surrounding hits in the detector from charged hadrons, neutral hadrons and photon energy which could contribute to a mismeasurement of their momentum. Only charged hadrons which are consistent with the signal primary vertex are considered in the calculation. As it is not possible to determine whether neutral hadrons are consistent with the signal primary vertex we can use the fact that the ratio of neutral to charged energy has been measured to be $\approx 0.5$. Hence the neutral hadronic energy coming from the primary vertex can be calculated as seen in Eqn.~\ref{eqn:neutralE}, where $E_{T,sub-leading}^{Charged Hadronic}$ is the transverse energy from charged hadrons which are associated to a sub-leading primary vertex. This is called the \emph{delta Beta correction}. The max() function ensures that the corrected neutral hadronic energy is never defined as negative.

\fixme{shorten variables names?}

\begin{centering}
\begin{equation}
\Sigma E_{T}^{Corrected Neutral Hadronic}  =  max(0, \Sigma E_{T}^{Neutral Hadronic} - 0.5*\Sigma E_{T,sub-leading}^{Charged Hadronic} )
\label{eqn:neutralE}
\end{equation}
\end{centering}



The RelIso formula with delta Beta correction can be found in~\ref{eqn:dBRelIso}. It is defined in a cone of radius $\textrm{R}=0.4$ and scaled by $1\; / \;\pt^{\mu}$ so that lower momemtum muons are required to have less energy from hadrons and photons in the cone to be considered isolated.

\begin{centering}
\begin{equation}
RelIso = \left( \Sigma E_{T}^{Charged Hadronic} + \Sigma E_{T}^{Corrected Neutral Hadronic} +  \Sigma E_{T}^{Photons} \right) \; / \;   \pt^{\mu}
\label{eqn:dBRelIso}
\end{equation}
\end{centering}


Formula~\ref{eqn:dBRelIso} is used to define relative isolation for muons; for electrons it is defined however the delta Beta correction is replaced by a rho correction, defined in equation~\ref{eqn:rhoCorr} where EA denotes Effective Area. [EA: derived from the event-specific average pile-up energy density per unit area in the phi-eta plane (rho) and the effective area based on shower shapes that the EGM POG has measured (depends on supercluster eta).] \fixme{why EA + reference}

\begin{centering}
\begin{equation}
rhoCorr = \left( \Sigma E_{T}^{Neutral Hadronic} + \Sigma E_{T}^{Photons} - \rho\times\textrm{EA} \right) 
\label{eqn:rhoCorr}
\end{equation}
\end{centering}


\begin{centering}
\begin{equation}
RelIso = \left( \Sigma E_{T}^{Charged Hadronic} + rhoCorr \right) \; / \;   \pt^{e}
\label{eqn:eRelIso}
\end{equation}
\end{centering}

\section{Muons \label{sec:muonreco}}

It is important to be able to identify isolated muons coming from the signal process rather than further decays from within jets or mismatched tracks. Applying the identification criteria in table~\ref{tab:muon_tight_cuts} can help to ensure a high purity of real muons are selected from the PF candidates described in section~\ref{sec:PF}. Two working points (WP) are defined, tight and loose. Tight muons have tighter requires on various quantities including \pt, as lower momentum muons are harder to identify. Tight muons are used when selecting how many muons are in the event. Loose muons are used to veto additional objects which are still likely to be muons but could be misidentified object such as pions. The loose criteria will capture more real muons in it's selection but with a lower purity but this is necessary when requiring a maximum number of muons in the event. The cut values for each working point are given in table~\ref{tab:muon_tight_cuts} for the 8 TeV analysis which is in section~\ref{c:Run1} and the 13 TeV analysis which is in section~\ref{c:Run2}.

The transverse impact parameter in the phi plane with respect to the leading primary vertex is denoted as d0 in the table. The distance between the leading primary vertex and the muon track in the z-direction is denoted dz. These two variable can be used to establish how consistent the muon track is with the leading primary vertex.

\fixme{Define other variables in table}

\begin{table}[htpb!]
\footnotesize
\begin{center}
\begin{tabular}{|r|c|c|c|c|}
\hline
\multicolumn{1}{|l|}{}                                          & \multicolumn{2}{c|}{Tight WP} & \multicolumn{2}{c|}{Loose WP} \\ \cline{2-5} 
Requirements                                                    & 8 TeV         & 13 TeV        & 8 TeV         & 13 TeV        \\ \hline
Is a GlobalMuon and a TrackerMuon                               & yes           & yes           & yes           & yes           \\
\pt (\GeV) $>$                                                  & 30            & 26            & 10            & 10            \\
$\lvert \eta \rvert <$                                          & 2.1           & 2.1           & 2.5           & 2.5           \\
RelIso, $<$                                                     & 0.12          & 0.15          & 0.2           & 0.25          \\
Number of valid hits in the tracker $>$                         & 5             & 5             & -             & -             \\
Number of hits in the muon stations $>$                         & 0             & 0             & -             & -             \\
d0 (cm) $<$ & 0.2           & 0.2           & -             & -             \\
dz (cm) $<$       & 0.5           & 0.5           & -             & -             \\
Number of pixel hits $>$                                        & 0             & 0             & -             & -             \\
Normalised $\chi^{2}$ of track $<$                              & 10            & 10            & -             & -             \\
Number of matched muon stations $>$                             & 1             & 1             & -             & -             \\ \hline
\end{tabular}
\caption{The cuts used for the tight and loose muon identification at 8 TeV and 13 TeV}
\label{tab:muon_tight_cuts}
\end{center}
\end{table}
\fixme{references for table}



\section{Electrons \label{sec:electronreco}}

Electrons lose a lot of energy in the tracker material via Bremsstrahlung. This is one of the biggest challenges in reconstructing electrons as this radiation needs to be taken into account to accurately measure their momentum and to have a good momentum resolution. These Bremsstrahlung photons can also convert into electron-positron pairs in the tracker material creating secondary electrons which must be distinguished from the signal electrons coming from the hard process. In this thesis the word electron is mostly used to include both charges, electron and positron. 

As with section~\ref{sec:muonreco} for muon reconstruction, two working points are defined for electrons from the PF candidates. Tight electrons are used when requiring electrons as part of the signal process and a looser set of criteria are used to in order to veto on extra electrons in the event to have a strict selection. The loose selection will of course contain more electrons but with a lower efficiency. 

There are multiple ways to identify electrons, two of which are used in this thesis. At 8 TeV a multivariarte technique to identify electrons was used. At 13 TeV a \emph{cuts based identification} was used as the electron tools from CMS were not as advanced by the time the analysis was performed.

The tight and veto working points at 8 TeV can be found in table~\ref{tab:electron_tight_cuts8} and are found for the barrel and endcap in table~\ref{tab:electron_tight_cuts13} for the 13 TeV analysis. 


\begin{table}[htpb!]
\footnotesize
\begin{center}
\begin{tabular}{|r|c|c|}
\hline
Requirements   & Tight & Veto \\ \hline
\ET $>$ GeV & 30    & 20   \\
$|\eta| <$  & 2.5   & 2.5  \\
$|$d$0| < $ & 0.02  & -    \\
ConversionVeto & yes   & -    \\
MVA ID $>$  & 0.9   & 0    \\
RelIso $<$  & 0.1   & 0.2  \\ \hline
\end{tabular}
\caption{The cuts used for the tight and veto electron identification at 8 TeV}
\label{tab:electron_tight_cuts8}
\end{center}
\end{table}

\begin{table}[htpb!]
\footnotesize
\begin{center}
\begin{tabular}{|r|c|c|c|c|}
\hline
& \multicolumn{2}{c|}{Tight} & \multicolumn{2}{c|}{Veto} \\
\cline{2-5}
%&Tight & TIght & Veto & Veto \\
Requirements &  Barrel        &   Endcap  &  Barrel        &   Endcap  \\
%&  ($|\eta_{SuCluster}|< 1.4442$)         &   ($1.5660<|\eta_{SuCluster}|<2.5$)  &  ($|\eta_{SuCluster}|< 1.4442$)         &   ($1.5660<|\eta_{SuCluster}|<2.5$)  \\

\hline
full $5\times5 \; \sigma_{I_{\eta}I_{\eta}} < $ & 0.0101 & 0.0279 & 0.0114 & 0.0352\\
$|\Delta \eta_{In}| < $  & 0.00926 & 0.00724  & 0.0152 & 0.0113  \\
$|\Delta \phi_{In}| < $  &  0.0336 & 0.0918 &  0.216 & 0.237  \\
$\frac{h}{E} <$ &0.0597 & 0.0615  &0.181 & 0.116  \\
relIso with rho correction  $\leq$  & 0.0354 & 0.0646& 0.126 & 0.144\\
$\frac{1}{E} - \frac{1}{p} < $ & 0.012 & 0.00999  & 0.207 & 0.174 \\
$|$d$0| < $  & 0.0111 & 0.0351  & 0.0564 & 0.222\\
$|$dz$| < $  & 0.0466 & 0.417 & 0.472 & 0.921\\
expected missing inner hits $\leq$ & 2 & 1 & 2 & 3  \\
pass conversion veto & yes & yes& yes & yes  \\
\hline
\end{tabular}
\caption{The cuts used for the tight and veto electron identification at 13 TeV where barrel is $|\eta_{SuCluster}|< 1.4442$ and endcap is  ($1.5660<|\eta_{SuCluster}|<2.5$)}
\label{tab:electron_tight_cuts13}
\end{center}
\end{table}

\fixme{add reference for electron cuts}
\fixme{transition region}
\fixme{add MVA electrons for 8 TeV}


\section{Jets \label{sec:jetreco}}
When partons such as quarks and gluons fragment and hadronise, predominantly into charged and neutral hadrons, they form showers of particles in approximately the same direction of travel as the original parton. These final state particles can be clustered into what is known as a \emph{jet} using the anti-$\kappa_{\textrm{T}}$ reconstruction algorithm~\cite{Cacciari:2008gp}. This is an infrared and collinear safe algorithm which starts with a high \pt `seed' hit in the calorimeter and uses the distance measure in equation~\ref{eqn:antikt1} to find the nearest hit to merge with. If the distance from the beam, $d_{iB}$ in equation~\ref{eqn:antikt2}, is smaller than the distance to another hit, $d_{ij}$, the particle is merged with the beam~\cite{Salam2010}.  For the analyses in this thesis a cone radius of R = 0.4 is used to reconstructed jet objects. 


\begin{equation}
d_{ij}=min\left( p_{\textrm{T}i}^{-2},p_{\textrm{T}j}^{-2} \right) \frac{\Delta R_{ij}^{2}}{R^{2}} \textrm{ , where } \Delta R_{ij}^{2} = {\left( y_{i} - y_{j}\right)}^{2} +  {\left( {\phi}_{i} - {\phi}_{j}\right)}^{2}
\label{eqn:antikt1}
\end{equation}

\begin{equation}
d_{iB}=p_{\textrm{T}i}^{2}
\label{eqn:antikt2}
\end{equation}

Corrections are applied to the jet energy to account for the non-uniform response of the detector in \pt and $\eta$. These are known as the \fixme*{ref and check if different between 8 \& 13}{L1FastJet and L2L3Residual}. The jet energy resolution (JER) is also smeared by 10$\%$ as the resolution is worse in data than in simulation. Loose jet identification critera are applied to suppress fake jets arising from electrons showering in the ECAL due to Bremsstrahlung. This includes requiring $|\eta|<2.5$, \pt$<30$ GeV and a separation from the nearest loose muon or electron of $\Delta\textrm{R}>0.4$.

\section{b-tagging ~\label{sec:btagreco}}
Reconstructing jets gives us the knowledge that the particles emerging from the collision include quarks and gluons. Being able to identify which flavour of quark showered in the detector is extremely useful for a wide range of analyses. Particularly for searches for final states containing four top quarks, the ability to identify b-quarks originating from the decay of top quarks is incredibly beneficial in allowing us to discriminate between the signal and irreducible backgrounds. 

The particle shower coming from the hadronisation of b-quarks will contain B mesons which travel further in the detector due to having longer decay times than light flavour mesons (u,d,s,g) resulting in a typical flight distance of up to a few centimetres~\cite{Collaboration2015BS0}. The \emph{impact parameter} (IP), defined as the distance between the primary vertex and the extrapolated point of closest approach of a track, will be larger for tracks coming from the decay of a B meson. The tracks emerging from this decay can form a secondary vertex. This information is exploited in the Combined Secondary Vertex (CSV) algorithm. The CSV algorithm is used to identify or \emph{tag} jets which originate from b-quarks by assigning a discriminator value between 0 and 1, where larger values are more consistent with b-quark jets. Loose (CSVL), medium (CSVM) and tight (CSVT) working points are defined at values of the discriminator for a given mis-identification rate. For analyses at $\sqrt{s} = 13$~TeV the algorithm was improved and is called Combined Secondary Vertex version 2 algorithm (CSVv2). Working points, selection efficiencies for b-quarks and mis-identification rates are gien in table~\ref{tab:btag} 

% \footnote{This efficiency was measured using the TTJets MLM sample for the medium working point at $\sqrt{13}$ TeV. The CSVv2L and CSVv2T efficiences have been taken from here~\cite{btageff} }.

\begin{table}[htpb!]
\footnotesize
\begin{center}
\begin{tabular}{l|l|c|c|c}
$\sqrt{s}$ (TeV)    & Name   & \multicolumn{1}{l|}{WorkingPoint} & \multicolumn{1}{l|}{Selection Efficiency ($\%$)} & \multicolumn{1}{l}{Mis-identification ($\%$)} \\ \hline
\multirow{3}{*}{8}  & CSVL   & 0.244                             &                                                  &                                               \\ \cline{2-5} 
                    & CSVM   & 0.679                             &                                                  & $\approx 1$                                   \\ \cline{2-5} 
                    & CSVT   & 0.898                             &                                                  &                                               \\ \hline
\multirow{3}{*}{13} & CSVv2L & 0.46                              & 82                                               & 11.5                                          \\ \cline{2-5} 
                    & CSVv2M & 0.8                               & 67                                               & 1.4                                           \\ \cline{2-5} 
                    & CSVv2T & 0.935                             & 47                                               & 0.15                                         
\end{tabular}
\caption{b-tagging working points and their selection and mistagging efficiencies}
\label{tab:btag}
\end{center}
\end{table}

Algorithms are currently in development to tag c-quark jets and to be able to identify the charge of b-quark jets (ref).

\section{Missing transverse energy ~\label{sec:METreco}}
As it is not possible to detect neutrinos and potentially some BSM particles because they interact so weakly with matter, we can infer their existence by examining the sum of the momentum of particles in the transverse plane of the detector, where the transverse plane is defined to be transverse to the beamline. We start with the assumption that the total momentum in the transverse plane is zero. An imbalance in the sum of the momentum of detectable particles is considered to be missing transverse energy (\ETmiss), as defined in equation~\ref{eqn:MET}.
\begin{equation}
\ETmiss = ~- \sum_{\textrm{all particles,}i} \hat{ {p}_{\textrm{T},i} }
\label{eqn:MET}
\end{equation}



