\chapter{Theory}
\label{c:theory}

\section{Standard Model}

The standard model (SM) is a theory which describes the fundamental particles and their interactions. Matter consists of six quarks and six leptons, each of which has an anti-particle with opposite sign quantum numbers. They are organised into three generations each of which contains heavier particles than the last as seen in Table~\ref{table:SMmatter}. All of the stable matter of universe is made up of particles from the first generation. The leptonic sector consists of charged leptons, which can interact via the electromagnetic and weak forces, and neutrinos, which interact via the weak force. The neutrinos are assumed to be massless in the standard model however observations of neutrino oscillations revealed that neutrinos have mass.

\begin{table}[ht!]
\centering
\caption{Quarks and leptons}
\label{table:SMmatter}
\footnotesize
\begin{tabular}{|c|l|l|l|l|l|l|}
\hline
\multirow{2}{*}{Generation} & \multicolumn{3}{c|}{Quarks}                             & \multicolumn{3}{c|}{Leptons}              \\ \cline{2-7} 
                            & Flavour & Charge & Mass (MeV)                           & Flavour      & Charge & Mass (MeV)        \\ \hline
\hline

\multirow{2}{*}{I}          & u       & 2/3    & $2.2^{+0.6}_{-0.4}$                  & e            & -1     & 0.511             \\
                            & d       & -1/3   & $4.7^{+0.5}_{-0.4}$                  & $\nu_{e}$    & 0      & $<2\times10^{-6}$ \\ \hline
\multirow{2}{*}{II}         & c       & 2/3    & $(1.27\pm 0.03)\times10^{3}$         & $\mu$        & -1     & 105.66            \\
                            & s       & -1/3   & $96^{+8}_{-4}$                       & $\nu_{\mu}$  & 0      & $<0.19$           \\ \hline
\multirow{2}{*}{III}        & t       & 2/3    & $(173.21\pm0.51\pm0.71)\times10^{3}$ & $\tau$       & -1     & $1776.86\pm0.12$  \\
                            & b       & -1/3   & $(4.18^{+0.04}_{-0.3})\times10^{3}$  & $\nu_{\tau}$ & 0      & $<18.2$           \\ \hline
\end{tabular}
\end{table}

Quarks interact via the electromagnetic or strong force. Each quarks carries a colour charge of red, green or blue, where all three colours combined give a colourless combination, or a colour and it's anti-colour can give a colourless combination. This theory of colour confinement means that quarks can only be found in bound states of baryons or mesons, respectively. 
The top quarks is the heaviest quark with a mass approximately equivalent to a lead nuclei. The top quark is the only quark which predominantly decays before it hadronises due to it's short lifetime of $5\times10^{-25}$~seconds. The main decay mode for top quarks is to a bottom quark and a W boson which occurs $95.6\pm3.4\%$ of the time.
Finally, the force carries consist of gauge bosons of integer spin, as seen in table~\ref{table:SMbosons}. Photons and Z bosons can mediate neutral electroweak interactions whereas W bosons can mediate charged electroweak interactions. The gluons mediate the strong interaction and occur with 8 different types of colour charge which will be described in section~\ref{subsec:QCD}. 
The Graviton is hypothesised to carry the gravitational force but there is as yet no evidence to support this hypothesis.

The discovery of the Higgs boson in 2012 completed the standard model with an explanation to how the fundamental particle have mass vie the electroweak symmetry breaking mechanism.


\begin{table}[ht!]
\centering
\caption{Gauge bosons}
\footnotesize
\label{table:SMbosons}
\begin{tabular}{|l|l|l|l|l|l|}
\hline
Gauge boson                       & Force           & Charge & Mass (GeV) & Spin & Range (m)  \\ \hline \hline
Photon ($\gamma$)                 & electromagnetic & 0      & 0          & 1    & $\infty$   \\ \hline
W$^{\pm}$                         & weak            & $\pm1$ &            & 1    & $10^{-18}$ \\ \hline
Z                                 & weak            & 0      &            & 1    & $10^{-18}$ \\ \hline
gluon                             & strong          & 0      & 0          & 1    & $10^{-15}$ \\ \hline
Graviton\footnote\{hypothesised\} & gravitational   & 0      & 0          & 2    & $\infty$   \\ \hline
\end{tabular}
\end{table}

\subsection{Quantum electrodynamics}

\subsection{Quantum chromodynamics}
\label{subsec:QCD}

\section{Four top quark production}

\section{BSM models with four top quark signatures}