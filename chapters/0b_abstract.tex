\label{c:abstract}
\chapter*{Abstract} 
% The results of the search for four top quark production are presented...{
\small{
The Standard Model (SM) of Particle Physics has been incredibly successful and accurate in describing the fundamental particles that make up the world around us and the way they behave. However, we know it is not the ultimate theory of nature as some phenomena remain unexplained. Outstanding questions include what is dark matter, this mysterious material which can be inferred from observations of the universe but has not been directly detected; and where does gravity fit into the picture? These questions motivate our search for physics Beyond the Standard Model (BSM) at the Large Hadron Collider at the CERN research facility. Here we accelerate protons around a 27 km ring and collide them at four experiments located underground. When we collide protons we effectively collide the more fundamental particles within the protons, quarks and gluons.

This thesis focuses on research undertaken at the CMS experiment studying the heaviest quarks, top quarks, which are not found in nature but instead are produced in high-energy experiments. Top quarks are most often produced in pairs, however this thesis focuses on the search for the simultaneous production of four top quarks, which is an incredibly rare process in comparison. A precision measurement of this rare process would be a stringent test on the SM and may give hints of physics beyond the standard model. 
Untangling the signal of four-top-quark production from the overwhelming background of top-quark-pair production in the output of the detector is incredibly difficult. Algorithms, which are often used in developing artificial intelligence, are therefore employed to exploit subtle differences in signatures, greatly increasing the sensitivity.
Results are presented which place tight limits on the rate of four-top-quark production, and projections of the future sensitivity are made including an estimate of when CMS will have sufficient data to definitively observe this process at SM rates. The results also allow us to place constraints on properties of hypothesised BSM particles. Here we interpret the results to place constraints on the mass and top quark-coupling of one such particle, the sgluon.}

\clearpage
\chapter*{Samenvatting}
\footnotesize{ Het Standaard Model (SM) van deeltjesfysica is ongelofelijk succesvol en nauwkeurig in het beschrijven van de fundamentele deeltjes in de wereld om ons heen en de manier waarop ze zich gedragen. Maar we weten dat het niet de ultieme natuurkundige theorie is, omdat een aantal verschijnselen onverklaard blijven in het SM. Openstaande vragen zijn: wat is donkere materie, het bestaan van deze mysterieuze substantie kan worden afgeleid uit waarnemingen van het heelal, maar is nog niet rechtstreeks waargenomen; en waar past de zwaartekracht in het plaatje? Deze vragen motiveren onze zoektocht naar de fysica voorbij het Standaard Model (BSM) bij de Large Hadron Collider op het CERN. Hier versnellen we protonen rond een ring van 27 km omtrek. Op vier plaatsen aan de ring liggen ondergrondse experimenten waar de protonen worden gebotst. Wanneer protonen botsen, bestuderen we effectief de werkelijk fundamentele deeltjes in het proton, quarks en gluonen. 
Dit proefschrift richt zich op onderzoek gebruikmakend van de proton-proton botsingen waargenomen door het CMS-experiment. Het zwaarste bekende elementaire deeltje, de top-quark, is niet te vinden in de natuur, maar in plaats daarvan kan worden geproduceerd in de botsingen bij de LHC. Top quarks worden meestal geproduceerd in paren, maar dit proefschrift richt zich op de zoektocht naar de productie van vier top-quark tegelijkertijd, dat is een ongelooflijk zeldzaam proces in vergelijking met paarproductie. Een nauwkeurige meting van deze zeldzame proces zou een strenge test zijn van het SM en kan hints geven of er nieuwe deeltjes worden gemaakt samen die in vier top quarks uiteenvallen. Het identificeren van het signaal van vier top-quark productie in de overweldigende achtergrond van top-quark-paarproductie in de data is een ongelooflijk moeilijke wetenschappelijke uitdaging. Hiervoor worden machine-learning algoritmen toegepast. Op deze wijze kunnen subtiele verschillen tussen de productie van top quark paren en vier top quarks in de botsing worden benut, en dit leidt tot een aanzienlijke verhoging van de gevoeligheid van de data-analyse. 
De resultaten van dit onderzoek plaatsen de meest strakke grenzen aan de werkzame doorsnede voor productie van vier top quarks. De resultaten zijn ook in staat om beperkingen te geven op de eigenschappen van eventuele hypothetische BSM deeltjes. In dit onderzoek worden daarom de resultaten ook ge{\"i}nterpreteerd als een functie van de massa en top quark-koppeling van zo'n hypothetisch nieuw deeltje, het zogenaamde sgluon.}

\normalsize
