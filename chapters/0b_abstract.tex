\label{c:abstract}
\chapter*{Abstract} 
% The results of the search for four top quark production are presented...

The Standard Model (SM) of Particle Physics has been incredibly successful and accurate in describing the fundamental particles that make up the world around us and they way they behave. However, we know it is not the ultimate theory of nature as some phenomena remain unexplained. Outstanding questions include what is dark matter, this mysterious material which can be inferred from observations of the universe but has not been directly detected; and where does gravity fit into the picture? These questions motivate our search for physics Beyond the Standard Model (BSM) at the Large Hadron Collider at the CERN research facility. Here we accelerate protons around a 27 km ring and collide them at four experiments located underground. When we collide protons we effectively collide the more fundamental particles within the protons, quarks and gluons.
This thesis focuses on research undertaken at the CMS experiment studying the heaviest quarks, top quarks, which are not found in nature but instead are produced in high-energy experiments. Top quarks are most often produced in pairs, however this thesis focuses on the search for the simultaneous production of four top quarks, which is an incredibly rare process in comparison. A precision measurement of this rare process would be a stringent test on the SM and may give hints of physics beyond the standard model. 
Untangling the signal of four-top-quark production from the overwhelming background of top-quark-pair production in the output of the detector is incredibly difficult. Algorithms, which are often used in developing artificial intelligence, are therefore employed to exploit subtle differences in signatures, greatly increasing the sensitivity..
Results are presented which place tight limits on the rate of four-top-quark production, and projections of the future sensitivity are made including an estimate of when CMS will have sufficient data to definitively observe this process at SM rates. The results allow us to place constraints on properties of hypothesised BSM particles. Here we interpret the results to place constraints on the mass and top quark-coupling of one such particle, the sgluon.
